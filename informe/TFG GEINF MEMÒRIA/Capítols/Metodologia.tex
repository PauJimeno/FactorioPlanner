% Indicate the main file. Must go at the beginning of the file.
% !TEX root = ../main.tex

%----------------------------------------------------------------------------------------
% CHAPTER TEMPLATE
%----------------------------------------------------------------------------------------


\chapter{Metodologia} % Main chapter title

\label{Metodologia} % Change X to a consecutive number; for referencing this chapter elsewhere, use \ref{ChapterX}

%----------------------------------------------------------------------------------------
% SECCIÓ 1: METODOLOGIA
%----------------------------------------------------------------------------------------
Com s'ha esmentat anteriorment aquest projecte s'enfoca principalment en l'àmbit de la recerca i no en el desenvolupament i manteniment d'un producte així que no s'ha usat cap metodologia de treball estandarditzada com per exemple SCRUM.\\

El que s'ha fet ha sigut fer reunions quinzenals amb el tutor on s'ha exposat la feina feta els últims 15 dies, parlat si calia modificar alguna part de la feina feta i s'ha plantejat la feina que s'havia de fer el següent Sprint de quinze dies, que principalment ha sigut implementar les restriccions que codifiquen cadascuna de les mecàniques del joc, fer una petita documentació per poder posar en context el funcionament al tutor i dur a terme petits testos sobre les restriccions implementades per comprovar-ne el funcionament.\\

Aquesta principalment ha sigut la metodologia tot i que algunes setmanes fos per temes personals, perquè alguna de les tasques del Sprint es compliqués o bé perquè em devies desenvolupar la interfície web, alguns Sprints s'han hagut d'allargar tres setmanes o per algun Sprint no s'ha pogut desenvolupar tot el que s'havia plantejat.\\
Aquesta metodologia dels Sprints s'ha seguit fins al final del projecte passant per tots els objectius principals plantejat a l'inici del projecte: desenvolupament del model bàsic, aplicació de millores i optimitzacions, desenvolupament de la interfície web i redacció de la memòria.\\

Al desenvolupament del projecte hi han participat en Mateu Villaret, que ha sigut el tutor i ha ajudat en proposar millores al model, aspectes de la redacció de la memòria, i en general a encarrilar el projecte cap al bon camí definint a cada reunió que és el més prioritari.\\
A més també s'ha comptat amb l'ajuda d'en Joan Espasa, que és un professor de la Universitat de Saint Andrews, co-autor de l'article \cite{arxivpaper}, que ha ajudat sobretot a introduir conceptes específics del funcionament del Z3 i ajuda amb el disseny d'algunes de les restriccions, ja que ell compta amb coneixement sobre el joc Factorio.


