% Indicate the main file. Must go at the beginning of the file.
% !TEX root = ../main.tex

%----------------------------------------------------------------------------------------
% CHAPTER TEMPLATE
%----------------------------------------------------------------------------------------


\chapter{Conclusions} % Main chapter title

\label{Conclusions} % Change X to a consecutive number; for referencing this chapter elsewhere, use \ref{ChapterX}
La resolució de problemes combinatoris ha sigut el que més m'ha cridat l'atenció del que hem après a la carrera i amb aquest treball només s'ha fet que reforçar aquest interès.\\
L'ambició inicial del projecte ha dut que algunes parts que es varen plantejar a l'inici no s'hagin pogut desenvolupar, aquestes parts, però no eren fonamentals pel projecte i es tractaven més aviat d'afegits que podien fer el treball més interessant. Al final, però no han impedit que el projecte hagi sortit rodó gràcies a la incorporació de la pàgina web que permet crear i visualitzar instàncies, cosa que ha fet que el projecte agafi un aspecte més ampli que el teòric. Aquesta decisió ha sigut molt important perquè m'ha ajudat a poder dividir el treball en dues parts i m'ha permès poder desconnectar d'una part mentre podia avançar en l'altre.\\

D'altra banda pel que fa als materials referents al projecte, he pogut veure com els problemes combinatoris són realment difícils i com alguns desenvolupaments de restriccions que d'entrada semblaven cosa fàcil, han desembocat en setmanes de feina. Això m'ha fet veure que especialitzar-se a saber modelar i optimitzar aquest tipus de problemes és una feina amb molt de valor i summament interessant.\\
També m'he adonat com les tecnologies SMT poden millorar el temps de resolució de problemes on és fonamental l'ús de nombres reals, que com s'ha pogut veure al final ha sigut una millora en fidelitat a les mecàniques del joc respecte a la implementació en EssencePrime de l'article publicat de St. Andrews.\\
Els resultats dels experiments han complert amb les expectatives que es van plantejar a l'inici del projecte, ja que s'han pogut resoldre instàncies complexes de mida 8x8. Tot i que respecte a l'article publicat de St. Andrews no s'han pogut resoldre instàncies més grans, cosa que hagués estat molt bé, per exemple arribar a resoldre instàncies de mida 10x10.